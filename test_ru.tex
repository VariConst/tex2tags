\documentclass[12pt]{article}
\usepackage[T2A]{fontenc}
\usepackage[utf8]{inputenc}
\usepackage[english,russian]{babel}
\usepackage{amsfonts,amsmath,amssymb,mathrsfs}
\usepackage[a4paper,top=20mm,bottom=15mm,left=20mm,right=15mm]{geometry}

\usepackage{hyphenat} % for \hyp{} command

\DeclareMathOperator{\arsh}{arsh}
%%%%%%%%%%%%%%%%%%%%%%%%%%%%%%%%%%%%%%%%%%%%%%%%%%%%%%%%%%%%%%%%%%%%%%%%%%%%%%%%
\begin{document}
%%%%%%%%%%%%%%%%%%%%%%%%%%%%%%%%%%%%%%%%%%%%%%%%%%%%%%%%%%%%%%%%%%%%%%%%%%%%%%%%
\title{\bf Тест программы tex2tags.py}
\author{А. А. Автор}
\date{}
{\let\newpage\relax\maketitle}
%%%%%%%%%%%%%%%%%%%%%%%%%%%%%%%%%%%%%%%%%%%%%%%%%%%%%%%%%%%%%%%%%%%%%%%%%%%%%%%%
\begin{abstract}
%%%%%%%%%%%%%%%%%%%%%%%%%%%%%%%%%%%%%%%%%%%%%%%%%%%%%%%%%%%%%%%%%%%%%%%%%%%%%%%%
Тест для программы\hyp{}помощника при переводе текстов, набранных с помощью
\LaTeX
%%%%%%%%%%%%%%%%%%%%%%%%%%%%%%%%%%%%%%%%%%%%%%%%%%%%%%%%%%%%%%%%%%%%%%%%%%%%%%%%
\end{abstract}
%%%%%%%%%%%%%%%%%%%%%%%%%%%%%%%%%%%%%%%%%%%%%%%%%%%%%%%%%%%%%%%%%%%%%%%%%%%%%%%%
\section{Раздел}
%%%%%%%%%%%%%%%%%%%%%%%%%%%%%%%%%%%%%%%%%%%%%%%%%%%%%%%%%%%%%%%%%%%%%%%%%%%%%%%%
Здесь идёт выключная формула с использованием \$\$\ldots\$\$
$$a = b + c.$$

Здесь идут формулы с использованием окружения \verb+subequations+
%
\begin{subequations}\label{eq1}
\begin{equation}
    \int x^p dx = \frac{x^{p+1}}{p+1},\quad p \ne -1,
\end{equation}
%
но
%
\begin{equation}
    \int \frac{dx}{x} = \ln x.
\end{equation}
\end{subequations}

Это просто формула в окружении \verb+equation+
%
\begin {equation}\label{eq2}
    e^{i\varphi} = \cos\varphi + i\sin\varphi,
\end {equation}
%
а это то же самое, но без номера формулы, т.\,е. \verb+equation*+
%
\begin{equation*}
    \sin i\varphi = i\sh\varphi.
\end{equation*}

Теперь \verb+eqnarray+
%
\begin{eqnarray}
    2x + 3y &=& 4, \\
    7x + 18y &=& -9.
\end{eqnarray}

Опять \$\$\ldots\$\$, но теперь \$\$ на отдельных строках
$$
    \arsh x = \ln(x + \sqrt{1 + x^2}).
$$

Ссылка на библиографию \verb+cite+ на использованную литературу\cite{ref1,ref2}
и ссылки \verb+ref+ на использованные формулы~(\ref{eq1}) и (\ref{eq2}).

Формула в тексте \$\ldots\$ на одной строке $\sin^2x + \cos^2x = 1$ и на
нескольких строках $\int\limits_{-\infty}^{+\infty} e^{-\alpha x^2} dx =
\sqrt{\frac{\pi}{\alpha}},\quad \int\limits_{0}^{\infty} t^{p - 1} e^{-t} dt
\equiv \Gamma(p) $.

%%%%%%%%%%%%%%%%%%%%%%%%%%%%%%%%%%%%%%%%%%%%%%%%%%%%%%%%%%%%%%%%%%%%%%%%%%%%%%%%
\begin{thebibliography}{9}
%%%%%%%%%%%%%%%%%%%%%%%%%%%%%%%%%%%%%%%%%%%%%%%%%%%%%%%%%%%%%%%%%%%%%%%%%%%%%%%%
\bibitem{ref1}
    Л. Д. Ландау, Е. М. Лифшиц, \textit{Механика}, Наука, Москва (1988).
\bibitem{ref2}
    Л. Д. Ландау, Е. М. Лифшиц, \textit{Теория поля}, Наука, Москва (1988).
%%%%%%%%%%%%%%%%%%%%%%%%%%%%%%%%%%%%%%%%%%%%%%%%%%%%%%%%%%%%%%%%%%%%%%%%%%%%%%%%
\end{thebibliography}
%%%%%%%%%%%%%%%%%%%%%%%%%%%%%%%%%%%%%%%%%%%%%%%%%%%%%%%%%%%%%%%%%%%%%%%%%%%%%%%%
\end{document}
%%%%%%%%%%%%%%%%%%%%%%%%%%%%%%%%%%%%%%%%%%%%%%%%%%%%%%%%%%%%%%%%%%%%%%%%%%%%%%%%
